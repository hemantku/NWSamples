% CS631 Advanced Programming in the UNIX Environment
% Author: Jan Schaumann <jschauma@netmeister.org>
% $Id: slides.tex,v 1.5 2005/09/13 03:13:26 jschauma Exp $
\special{! TeXDict begin /landplus90{true}store end }

\documentclass[xga]{xdvislides}
\usepackage[landscape]{geometry}
\usepackage{graphics}
\usepackage{graphicx}
\usepackage{colordvi}

\begin{document}
\setfontphv

%%% Headers and footers
\lhead{\slidetitle}
\chead{CS631 - Advanced Programming in the UNIX Environment}
\rhead{Slide \thepage}
\lfoot{\Gray{Lecture 03: Files and Directories}}
\cfoot{\relax}
\rfoot{\Gray{\today}}

\vspace*{\fill}
\begin{center}
	\Hugesize
		CS631 - Advanced Programming in the UNIX Environment\\
		-- \\
		Files and Directories
	\hspace*{5mm}\blueline\\ [1em]
	\Normalsize
		Department of Computer Science\\
		Stevens Institute of Technology\\
		Jan Schaumann\\
		\verb+jschauma@stevens.edu+\\
		\verb+http://www.cs.stevens.edu/~jschauma/631/+
\end{center}
\vspace*{\fill}

\subsection{Text Formatting 101}

This is a very long text block contnaing sentences with no re al
punctuation and no paragraphsText like this is hard to readAlsoreading
such lo ng text is no fun, and when the instructor is not having fun when
grading that might yield undesirable resluts for the studentEven though
this is a programming classit is i mportant to be able to write English
text well:programming is communicationand a ll communications requires
clarity.The majority of the work in any given program depend s on clear
communication of the requirementsof getting answers to questions and of wo
rking together with other peopleThereforeyou should practice writing
legigible and clear text that is easy to read and understandText should be
broken at around 80 charac ters and paragraphs should be created every now
and then to make it easier for the readerO therwise,you end up having just
this very long text block containing sentences w ith no real punctuation
and no paragraphsand as noted abovetext like that is hard to r eadno fun
and may lead to a grumpy instructorwhen you really want your instructor t
o be in a good mood when he is attempting to make sense of your homework
submissions.

\subsection{HW\#1 - some stats}
\begin{verbatim}
   7 pwd
   5 mkdir
   5 echo
   4 kill
   4 cp
   2 test
   2 mv
   2 cat
   1 rm
   1 ls
   1 ln
   1 date
   1 chmod
\end{verbatim}

\subsection{HW\#1 - some stats}

\begin{verbatim}
$ wc -w * | sort -n | column
      41 HW1-a          625 HW1-h        1295 HW1-n
     239 HW1-b          632 HW1-g        1359 HW1-o
     451 HW1-c          647 hw1-i        1409 HW1-p
     489 HW1-d          742 HW1-j        1750 HW1-q
     494 HW1-e          861 HW1-k
     497 HW1-f          965 HW1-l
     524 HW1-g         1079 HW1-m
$
\end{verbatim}


\subsection{When reading code...}

\vspace*{\fill}
\begin{center}
\Huge
First understand {\em what} it does. \\
\vspace{.5in}

Then understand {\em why} it does it. \\
\vspace{.5in}

Only then pay attention to {\em how} it does it.
\Normalsize
\end{center}
\vspace*{\fill}

\subsection{When reading code...}
\begin{itemize}
	\item compare source code and documentation; are they in sync?
	\item compare documentation and reality; are they in sync?
	\item review manual pages for system- and library calls made; are
		there (failure or success) cases unaccounted for?
	\item what follow-up questions do you have?
\end{itemize}

\subsection{"More code" does not necessarily imply a "better program".}

\vspace*{\fill}
\begin{center}
\Huge
``One of my most productive days was throwing away 1,000 lines of code.''
\Normalsize
\\

Ken Thompson
\end{center}
\vspace*{\fill}

\pagebreak

\vspace*{\fill}
\begin{center}
  \Hugesize
	Code Reading
	\hspace*{5mm}\blueline\\ [1em]
	HW\#2
  \Normalsize
\end{center}
\vspace*{\fill}



\subsection{{\tt stat(2)} family of functions}
\small
\setlength{\unitlength}{1mm}
\begin{center}
	\begin{picture}(150,40)
		\thinlines
		\put(0,0){\framebox(130,40){}}
		\put(10,35){{\tt \#include <sys/types.h>}}
		\put(10,30){{\tt \#include <sys/stat.h>}}
		\put(10,22){{\tt int stat(const char *{\em path}, struct stat *{\em sb});}}
		\put(10,18){{\tt int lstat(const char *{\em path}, struct stat *{\em sb});}}
		\put(10,14){{\tt int fstat(int {\em fd}, struct stat *{\em sb});}}
		\put(85,3){Returns: 0 if OK, -1 on error}
	\end{picture}
\end{center}
\Normalsize

All these functions return extended attributes about the referenced file (in
the case of {\em symbolic links}, {\tt lstat(2)} returns attributes of the
{\em link}, others return stats of the referenced file).

\subsection{{\tt stat(2)} family of functions}
\small
\setlength{\unitlength}{1mm}
\begin{center}
	\begin{picture}(150,40)
		\thinlines
		\put(0,0){\framebox(130,40){}}
		\put(10,35){{\tt \#include <sys/types.h>}}
		\put(10,30){{\tt \#include <sys/stat.h>}}
		\put(10,22){{\tt int stat(const char *{\em path}, struct stat *{\em sb});}}
		\put(10,18){{\tt int lstat(const char *{\em path}, struct stat *{\em sb});}}
		\put(10,14){{\tt int fstat(int {\em fd}, struct stat *{\em sb});}}
		\put(85,3){Returns: 0 if OK, -1 on error}
	\end{picture}
\end{center}
\Normalsize

All these functions return extended attributes about the referenced file (in
the case of {\em symbolic links}, {\tt lstat(2)} returns attributes of the
{\em link}, others return stats of the referenced file).
\vspace{.25in}
\small
\begin{verbatim}
struct stat {
    dev_t      st_dev;          /* device number (filesystem) */
    ino_t      st_ino;          /* i-node number (serial number) */
    mode_t     st_mode;         /* file type & mode (permissions) */
    dev_t      st_rdev;         /* device number for special files */
    nlink_t    st_nlink;        /* number of links */
    uid_t      st_uid;          /* user ID of owner */
    gid_t      st_gid;          /* group ID of owner */
    off_t      st_size;         /* size in bytes, for regular files */
    time_t     st_atime;        /* time of last access */
    time_t     st_mtime;        /* time of last modification */
    time_t     st_ctime;        /* time of last file status change */
    long       st_blocks;       /* number of 512-byte* blocks allocated */
    long       st_blksize;      /* best I/O block size */
};
\end{verbatim}
\Normalsize

\subsection{{\tt struct stat}: {\tt st\_mode}}

The {\tt st\_mode} field of the {\tt struct stat} encodes the type of file:

\begin{itemize}
	\item {\bf regular} -- most common, interpretation of data is up to
		application
	\item {\bf directory} -- contains names of other files and pointer to
		information on those files. Any process can read, only kernel
		can write.
	\item {\bf character special} -- used for certain types of devices
	\item {\bf block special} -- used for disk devices (typically). All
		devices are either {\em character} or {\em block special}.
	\item {\bf FIFO} -- used for interprocess communication
		(sometimes called {\em named pipe})
	\item {\bf socket} -- used for network communication and non-network
		communication (same host).
	\item {\bf symbolic link} -- Points to another file.
\end{itemize}
\vspace{.25in}
Find out more in {\tt <sys/stat.h>}.

\subsection{{\tt struct stat}: {\tt st\_mode}}
\begin{verbatim}
$ cc -Wall still-simple-ls.c
$ ./a.out /dev | more
\end{verbatim}

\subsection{{\tt struct stat}: {\tt st\_mode}, {\tt st\_uid} and {\tt st\_gid}}
Every process has six or more IDs associated with it:
\\

\begin{tabular}{| l | l |}
	\hline
	real user ID & who we really are \\
	real group ID & \\
	\hline
	effective user ID & used for file access permission checks \\
	effective group ID & \\
	supplementary group IDs & \\
	\hline
	saved set-user-ID & saved by {\tt exec} functions \\
	saved set-group-ID & \\
	\hline
\end{tabular}
\vspace{.25in}

Whenever a file is {\em setuid}, set the {\em effective user ID} to {\tt
st\_uid}. Whenever a file is {\em setgid}, set the {\em effective group ID} to
{\tt st\_gid}.  {\tt st\_uid} and {\tt st\_gid} always specify the owner and
group owner of a file, regardless of whether it is setuid/setgid.

\subsection{{\tt struct stat}: {\tt st\_mode}}
{\tt st\_mode} also encodes the file access permissions ({\tt S\_IRUSR},
{\tt S\_IWUSR}, {\tt S\_IXUSR}, {\tt S\_IRGRP}, {\tt S\_IWGRP}, {\tt S\_IXGRP},
{\tt S\_IROTH}, {\tt S\_IWOTH}, {\tt S\_IXOTH}).  Uses of the permissions are
summarized as follows:

\begin{itemize}
	\item To open a file, need execute permission on each directory component of the path
\end{itemize}

\subsection{{\tt struct stat}: {\tt st\_mode}}
{\tt st\_mode} also encodes the file access permissions ({\tt S\_IRUSR},
{\tt S\_IWUSR}, {\tt S\_IXUSR}, {\tt S\_IRGRP}, {\tt S\_IWGRP}, {\tt S\_IXGRP},
{\tt S\_IROTH}, {\tt S\_IWOTH}, {\tt S\_IXOTH}).  Uses of the permissions are
summarized as follows:

\begin{itemize}
	\item To open a file, need execute permission on each directory component of the path
	\item To open a file with {\tt O\_RDONLY} or {\tt O\_RDWR}, need read permission
\end{itemize}

\subsection{{\tt struct stat}: {\tt st\_mode}}
{\tt st\_mode} also encodes the file access permissions ({\tt S\_IRUSR},
{\tt S\_IWUSR}, {\tt S\_IXUSR}, {\tt S\_IRGRP}, {\tt S\_IWGRP}, {\tt S\_IXGRP},
{\tt S\_IROTH}, {\tt S\_IWOTH}, {\tt S\_IXOTH}).  Uses of the permissions are
summarized as follows:

\begin{itemize}
	\item To open a file, need execute permission on each directory component of the path
	\item To open a file with {\tt O\_RDONLY} or {\tt O\_RDWR}, need read permission
	\item To open a file with {\tt O\_WRONLY} or {\tt O\_RDWR}, need write permission
\end{itemize}

\subsection{{\tt struct stat}: {\tt st\_mode}}
{\tt st\_mode} also encodes the file access permissions ({\tt S\_IRUSR},
{\tt S\_IWUSR}, {\tt S\_IXUSR}, {\tt S\_IRGRP}, {\tt S\_IWGRP}, {\tt S\_IXGRP},
{\tt S\_IROTH}, {\tt S\_IWOTH}, {\tt S\_IXOTH}).  Uses of the permissions are
summarized as follows:

\begin{itemize}
	\item To open a file, need execute permission on each directory component of the path
	\item To open a file with {\tt O\_RDONLY} or {\tt O\_RDWR}, need read permission
	\item To open a file with {\tt O\_WRONLY} or {\tt O\_RDWR}, need write permission
	\item To use {\tt O\_TRUNC}, must have write permission
\end{itemize}


\subsection{{\tt struct stat}: {\tt st\_mode}}
{\tt st\_mode} also encodes the file access permissions ({\tt S\_IRUSR},
{\tt S\_IWUSR}, {\tt S\_IXUSR}, {\tt S\_IRGRP}, {\tt S\_IWGRP}, {\tt S\_IXGRP},
{\tt S\_IROTH}, {\tt S\_IWOTH}, {\tt S\_IXOTH}).  Uses of the permissions are
summarized as follows:

\begin{itemize}
	\item To open a file, need execute permission on each directory component of the path
	\item To open a file with {\tt O\_RDONLY} or {\tt O\_RDWR}, need read permission
	\item To open a file with {\tt O\_WRONLY} or {\tt O\_RDWR}, need write permission
	\item To use {\tt O\_TRUNC}, must have write permission
	\item To create a new file, must have write+execute permission for the directory
\end{itemize}


\subsection{{\tt struct stat}: {\tt st\_mode}}
{\tt st\_mode} also encodes the file access permissions ({\tt S\_IRUSR},
{\tt S\_IWUSR}, {\tt S\_IXUSR}, {\tt S\_IRGRP}, {\tt S\_IWGRP}, {\tt S\_IXGRP},
{\tt S\_IROTH}, {\tt S\_IWOTH}, {\tt S\_IXOTH}).  Uses of the permissions are
summarized as follows:

\begin{itemize}
	\item To open a file, need execute permission on each directory component of the path
	\item To open a file with {\tt O\_RDONLY} or {\tt O\_RDWR}, need read permission
	\item To open a file with {\tt O\_WRONLY} or {\tt O\_RDWR}, need write permission
	\item To use {\tt O\_TRUNC}, must have write permission
	\item To create a new file, must have write+execute permission for the directory
	\item To delete a file, need write+execute on directory, file doesn't matter
\end{itemize}


\subsection{{\tt struct stat}: {\tt st\_mode}}
{\tt st\_mode} also encodes the file access permissions ({\tt S\_IRUSR},
{\tt S\_IWUSR}, {\tt S\_IXUSR}, {\tt S\_IRGRP}, {\tt S\_IWGRP}, {\tt S\_IXGRP},
{\tt S\_IROTH}, {\tt S\_IWOTH}, {\tt S\_IXOTH}).  Uses of the permissions are
summarized as follows:

\begin{itemize}
	\item To open a file, need execute permission on each directory component of the path
	\item To open a file with {\tt O\_RDONLY} or {\tt O\_RDWR}, need read permission
	\item To open a file with {\tt O\_WRONLY} or {\tt O\_RDWR}, need write permission
	\item To use {\tt O\_TRUNC}, must have write permission
	\item To create a new file, must have write+execute permission for the directory
	\item To delete a file, need write+execute on directory, file doesn't matter
	\item To execute a file (via {\tt exec} family), need execute permission
\end{itemize}

\subsection{{\tt access(2)}}
\small
\setlength{\unitlength}{1mm}
\begin{center}
	\begin{picture}(150,25)
		\thinlines
		\put(0,0){\framebox(130,25){}}
		\put(10,20){{\tt \#include <unistd.h>}}
		\put(10,12){{\tt int access(const char *{\em path}, int {\em mode});}}
		\put(85,3){Returns: 0 if OK, -1 on error}
	\end{picture}
\end{center}
\Normalsize

Tests file accessibility on the basis of the {\em real} uid and gid. Allows
setuid/setgid programs to see if the real user could access the file without
it having to drop permissions to do so.
\\

The {\tt mode} paramenter can be a bitwise OR of:

\begin{itemize}
	\item {\tt R\_OK} -- test for read permission
	\item {\tt W\_OK} -- test for write permission
	\item {\tt X\_OK} -- test for execute permission
	\item {\tt F\_OK} -- test for existence of file
\end{itemize}

\subsection{{\tt access(2)}}
\begin{verbatim}
$ cc -Wall access.c
$ ./a.out /etc/passwd
access ok for /etc/passwd
open ok for /etc/passwd
$ ./a.out /etc/master.passwd
access error for /etc/master.passwd
open error for /etc/master.passwd
$ sudo chown root a.out
$ sudo chmod 4755 a.out
$ ./a.out /etc/passwd
access ok for /etc/passwd
open ok for /etc/passwd
$ ./a.out /etc/master.passwd
access error for /etc/master.passwd
open ok for /etc/master.passwd
$
\end{verbatim}

\subsection{{\tt access(2)}}
On Mac OS X:
\begin{verbatim}
$ ls -l a.out
-rwsr-xr-x  1 root  staff  8732 Sep 15 22:35 a.out
$ cc -Wall access.c
ld: can't write output file: a.out for architecture x86_64
clang: error: linker command failed with exit code 1 (use -v to see invocation)
$ ls -l a.out
-rwsr-xr-x  1 root  staff  8732 Sep 15 22:35 a.out
$ gcc -Wall access.c
ld: can't write output file: a.out for architecture x86_64
collect2: ld returned 1 exit status
$ ls -l a.out
ls: a.out: No such file or directory
\end{verbatim}

\subsection{{\tt access(2)}}
\begin{verbatim}
$ sudo dtruss gcc -Wall access.c
[...]
10157/0x48d34:  access("a.out\0", 0x0, 0x0)             = 0 0
10157/0x48d34:  access("a.out\0", 0x2, 0x0)             = -1 Err#13
10157/0x48d34:  unlink("a.out\0", 0x0, 0x100365AF0)     = 0 0
[...]
$ ls -ld . a.out
$ rm -f a.out
\end{verbatim}


\subsection{{\tt access(2)}}
On NetBSD:
\begin{verbatim}
$ ls -l a.out
-rwsr-xr-x  1 root  users  9399 Sep 26 16:12 a.out
$ cc -Wall access.c
$ ls -l a.out
-rwxr-xr-x  1 jschauma  users  9399 Sep 26 16:51 a.out
$ ktruss cc -Wall access.c
[...]
 14250      1 ld       CALL  __stat30(0x43a1bd,0x7f7fffffd340)
 14250      1 ld       NAMI  "a.out"
 14250      1 ld       CALL  unlink(0x43a1bd)
 14250      1 ld       NAMI  "a.out"
 14250      1 ld       CALL  open(0x43a1bd,0x602,0x1b6)
 14250      1 ld       NAMI  "a.out"
[...]
\end{verbatim}

\subsection{{\tt access(2)}}
On Linux:
\begin{verbatim}
$ ls -l a.out
-rwxr-xr-x 1 root users 6555 Sep 24 20:35 a.out
$ cc -Wall access.c
$ ls -l a.out
-rwxr-xr-x 1 jschauma users 6555 Sep 24 20:36 a.out
$ strace -f cc -Wall access.c
[...]
[pid 11721] stat("a.out", {st_mode=S_IFREG|0755, st_size=6555, ...}) = 0
[pid 11721] lstat("a.out", {st_mode=S_IFREG|0755, st_size=6555, ...}) = 0
[pid 11721] unlink("a.out")             = 0
[pid 11721] open("a.out", O_RDWR|O_CREAT|O_TRUNC, 0666) = 3
[...]
\end{verbatim}


\subsection{{\tt struct stat}: {\tt st\_mode}}
Which permission set to use is determined (in order listed):
\begin{enumerate}
	\item If effective-uid == 0, grant access
\end{enumerate}

\subsection{{\tt struct stat}: {\tt st\_mode}}
Which permission set to use is determined (in order listed):
\begin{enumerate}
	\item If effective-uid == 0, grant access
	\item If effective-uid == {\tt st\_uid}
		\begin{enumerate}
			\item if appropriate user permission bit is set, grant access
			\item else, deny access
		\end{enumerate}
\end{enumerate}

\subsection{{\tt struct stat}: {\tt st\_mode}}
Which permission set to use is determined (in order listed):
\begin{enumerate}
	\item If effective-uid == 0, grant access
	\item If effective-uid == {\tt st\_uid}
		\begin{enumerate}
			\item if appropriate user permission bit is set, grant access
			\item else, deny access
		\end{enumerate}
	\item If effective-gid == {\tt st\_gid}
		\begin{enumerate}
			\item if appropriate group permission bit is set, grant access
			\item else, deny access
		\end{enumerate}
\end{enumerate}

\subsection{{\tt struct stat}: {\tt st\_mode}}
Which permission set to use is determined (in order listed):
\begin{enumerate}
	\item If effective-uid == 0, grant access
	\item If effective-uid == {\tt st\_uid}
		\begin{enumerate}
			\item if appropriate user permission bit is set, grant access
			\item else, deny access
		\end{enumerate}
	\item If effective-gid == {\tt st\_gid}
		\begin{enumerate}
			\item if appropriate group permission bit is set, grant access
			\item else, deny access
		\end{enumerate}
	\item If appropriate other permission bit is set, grant access, else deny access
\end{enumerate}

\subsection{{\tt struct stat}: {\tt st\_mode}}
Ownership of {\bf new} files and directories:
\begin{itemize}
	\item {\tt st\_uid} = effective-uid
	\item {\tt st\_gid} = ...either:
		\begin{itemize}
			\item effective-gid of process
			\item gid of directory in which it is being created
		\end{itemize}
\end{itemize}

\subsection{{\tt umask(2)}}
\small
\setlength{\unitlength}{1mm}
\begin{center}
	\begin{picture}(150,25)
		\thinlines
		\put(0,0){\framebox(130,25){}}
		\put(10,20){{\tt \#include <sys/stat.h>}}
		\put(10,12){{\tt mode\_t umask(mode\_t {\em numask});}}
		\put(62,3){Returns: previous file mode creation mask}
	\end{picture}
\end{center}
\Normalsize

{\tt umask(2)} sets the file creation mode mask. Any bits that are {\em on} in
the file creation mask are turned {\em off} in the file's mode.
\\

Important because a user can set a default umask. If a program needs to be
able to insure certain permissions on a file, it may need to turn off (or
modify) the umask, which affects only the current process.


\subsection{{\tt umask(2)}}
\begin{verbatim}
$ cc -Wall umask.c
$ umask 022
$ touch foo
$ ./a.out
$ ls -l foo*
-rw-r--r--  1 jschauma  staff  0 Sep 26 18:35 foo
-rw-r--r--  1 jschauma  staff  0 Sep 26 18:36 foo1
-rw-rw-rw-  1 jschauma  staff  0 Sep 26 18:36 foo2
-rw-------  1 jschauma  staff  0 Sep 26 18:36 foo3
\end{verbatim}

\subsection{{\tt chmod(2)}, {\tt lchmod(2)} and {\tt fchmod(2)}}
\small
\setlength{\unitlength}{1mm}
\begin{center}
	\begin{picture}(150,35)
		\thinlines
		\put(0,0){\framebox(130,35){}}
		\put(10,30){{\tt \#include <sys/stat.h>}}
		\put(10,22){{\tt int chmod(const char *{\em path}, mode\_t {\em mode});}}
		\put(10,17){{\tt int lchmod(const char *{\em path}, mode\_t {\em mode});}}
		\put(10,12){{\tt int fchmod(int {\em fd}, mode\_t {\em mode});}}
		\put(85,3){Returns: 0 if OK, -1 on error}
	\end{picture}
\end{center}
\Normalsize

Changes the permission bits on the file. Must be either superuser or {\em
effective uid} == {\tt st\_uid}. {\em mode} can be any of the bits from our
discussion of {\tt st\_mode} as well as:
\begin{itemize}
	\item {\tt S\_ISUID} -- setuid
	\item {\tt S\_ISGID} -- setgid
	\item {\tt S\_ISVTX} -- sticky bit (aka ``saved text'')
	\item {\tt S\_IRWXU} -- user read, write and execute
	\item {\tt S\_IRWXG} -- group read, write and execute
	\item {\tt S\_IRWXO} -- other read, write and execute
\end{itemize}

\subsection{{\tt chmod(2)}, {\tt lchmod(2)} and {\tt fchmod(2)}}
\begin{verbatim}
$ rm foo*
$ umask 077
$ touch foo foo1
$ chmod a+rx foo
$ ls -l foo*
-rwxr-xr-x  1 jschaumann  staff  0 Sep 15 23:00 foo
-rw-------  1 jschaumann  staff  0 Sep 15 23:00 foo1
$ cc -Wall chmod.c
$ ./a.out
$ ls -l foo foo1
-rwsr--r-x  1 jschaumann  staff  0 Sep 15 23:01 foo
-rw-r--r--  1 jschaumann  staff  0 Sep 15 23:01 foo1
$
\end{verbatim}

\subsection{{\tt chown(2)}, {\tt lchown(2)} and {\tt fchown(2)}}
\small
\setlength{\unitlength}{1mm}
\begin{center}
	\begin{picture}(150,35)
		\thinlines
		\put(0,0){\framebox(130,35){}}
		\put(10,30){{\tt \#include <unistd.h>}}
		\put(10,22){{\tt int chown(const char *{\em path}, uid\_t {\em owner}, gid\_t {\em group});}}
		\put(10,17){{\tt int lchown(const char *{\em path}, uid\_t {\em owner}, gid\_t {\em group});}}
		\put(10,12){{\tt int fchown(int {\em fd}, uid\_t {\em owner}, gid\_t {\em group});}}
		\put(85,3){Returns: 0 if OK, -1 on error}
	\end{picture}
\end{center}
\Normalsize

Changes {\tt st\_uid} and {\tt st\_gid} for a file. For BSD, must be
superuser. Some SVR4's let users chown files they own. POSIX.1 allows either
depending on {\tt \_POSIX\_CHOWN\_RESTRICTED} (a kernel constant).
\\

{\em owner} or {\em group} can be -1 to indicate that it should remain the same.
Non-superusers can change the {\tt st\_gid} field if both:
\begin{itemize}
	\item effective-user ID == {\tt st\_uid} and
	\item {\em owner} == file's user ID and {\em group} == effective-group ID
		(or one of the supplementary group IDs)
\end{itemize}
\addvspace{.5in}
{\tt chown} and friends clear all setuid or setgid bits.

\subsection{Directory sizes (on a system using UFS)}
\begin{verbatim}
$ cd /tmp
$ mkdir -p /tmp/d
$ ls -ld /tmp/d
drwxr-xr-x  2 jschauma  wheel  512 Sep 26 19:35 /tmp/d
$ touch d/a d/b d/c d/d d/e d/f d/g
$ ls -ld /tmp/d
drwxr-xr-x  2 jschauma  wheel  512 Sep 26 19:35 /tmp/d
$ touch d/$(jot -b a 255 | tr -d '[:space:]')
$ ls -ld /tmp/d
drwxr-xr-x  2 jschauma  wheel  512 Sep 26 19:35 /tmp/d
$ touch d/$(jot -b b 255 | tr -d '[:space:]')
$ ls -ld /tmp/d
drwxr-xr-x  2 jschauma  wheel  1024 Sep 26 19:37 /tmp/d
$ rm /tmp/d/a*
$ ls -ld /tmp/d
drwxr-xr-x  2 jschauma  wheel  1024 Sep 26 19:37 /tmp/d
$
\end{verbatim}

\subsection{Directory sizes (on a system using HFS+)}
\begin{verbatim}
$ cd /tmp
$ mkdir -p /tmp/d
$ cd /tmp/d
$ ls -ld
drwxr-xr-x  2 jschauma  wheel  68 Sep 24 18:52 .
$ touch a
$ ls -ld
drwxr-xr-x  3 jschauma  wheel  102 Sep 24 18:52 .
$ echo $((102 / 3))
34
$ touch c
$ ls -ld
drwxr-xr-x  4 jschauma  wheel  136 Sep 24 18:52 .
$ rm c
$ ls -ld
drwxr-xr-x  3 jschauma  wheel  102 Sep 24 18:52 .
$
\end{verbatim}

\subsection{Homework}
Reading:
\begin{itemize}
	\item manual pages for the functions covered
	\item Stevens Chap. 4.1 through 4.13
\end{itemize}
Playing:
\begin{itemize}
	\item in your shell, set your umask to various values and see what
          happens to new files you create (example: Stevens \# 4.3)
	\item Verify that turning off user-read permission for a file that you own
		denies you access to the file, even if group- or other permissions
		allow reading.
\end{itemize}
\addvspace{.5in}
Midterm Assignment:
\verb+http://www.cs.stevens.edu/~jschauma/631/f13-midterm.html+

\end{document}
